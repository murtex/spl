
	% frequency domain
\subsection{Frequency domain}

\begin{frame}[fragile] % ----------------------------------------------------
	\frametitle{Frequency domain}
	\begin{itemize}
		\item \textbf{discrete Fourier transform}, time domain $\rightarrow$ frequency domain
			\begin{align*}
				X_k=\sum_{i=1}^Nx_i\,\ee^{-2\pi\ii\mfrac{(i-1)(k-1)}N}\in\mathbb C\quad\textrm{with}\quad k\in\{1,\ldots,N\}
			\end{align*}
			\begin{code}
>> Xk = fft( xi ) / N; \color{medium}% complex Fourier coefficients
			\end{code}
		\item $k$ is a \textbf{frequency index} (as $i$ was a time index)
			\begin{align*}
				k\rightarrow f_k=\frac{k-1}Nf_{\textrm S}
			\end{align*}
			\begin{code}
>> fk = (0:N-1) / N * fS; \color{medium}% frequency values
			\end{code}
		\item frequencies beyond Nyquist frequency are \textbf{negative frequencies}
			\begin{align*}
				f_k\rightarrow\begin{cases}f_k-f_{\textrm S}&\textrm{if $f_k>f_{\textrm{Ny}}$}\\f_k&\textrm{otherwise}\end{cases}
			\end{align*}
			\begin{code}
>> fk(fk > fNy) = fk(fk > fNy) - fS; \color{medium}% imply negative frequencies
			\end{code}
	\end{itemize}
\end{frame}

\begin{frame}[fragile] % ----------------------------------------------------
	\frametitle{Frequency domain}
	\begin{itemize}
		\item \textbf{power spectral density} (also known as \textbf{power spectrum})
			\begin{align*}
				P_k=\lvert X_k\rvert^2\in\mathbb R\quad\Leftarrow\quad\sum_{k=1}^NP_k=P
			\end{align*}
			\begin{code}
>> Pk = abs( Xk ) .^ 2; \color{medium}% power spectral density
			\end{code}
		\item \textbf{real valued signals} ($x_i\in\mathbb R$) imply a special symmetry
			\begin{align*}
				X_{f_k}=X_{-f_k}^*\quad\Rightarrow\quad P_{f_k}=P_{-f_k}
			\end{align*}
		\item restrict to \textbf{one-sided spectrum}
			\begin{code}
>> Pk(fk < 0) = []; \color{medium}% remove negative frequency components
>> Xk(fk < 0) = [];
>> fk(fk < 0) = [];
>> Pk(2:end) = 2 * Pk(2:end); \color{medium}% rescale to match total power
>> Xk(2:end) = sqrt( 2 ) * Xk(2:end);
			\end{code}
		\item $P_1$ is \textbf{DC offset}, $P_{k>1}$ are \textbf{contributions of sines} with frequencies $f_{k>1}$
			\begin{align*}
				x(t)=\sqrt{P_1}+\sqrt2\sum_{k>1}\sqrt{P_k}\sin(2\pi f_kt)
			\end{align*}
	\end{itemize}
\end{frame}

\begin{frame} % -------------------------------------------------------------
	\frametitle{Frequency domain}
	\begin{itemize}
		\item complex valued but without loss of phase information
			\begin{align*}
				x(t)=X_1+\sqrt2\sum_{k>1}X_k\ee^{2\pi\ii f_kt}
			\end{align*}
		\item \textbf{example}: \texttt{matlab/fdomain.m}
			\begin{figure}
				\centering
				\begin{subfigure}[c]{0.48\linewidth}
					\cfbox{neutral}{\includegraphics[width=\linewidth]{images/fdomain_decomp.eps}}
				\end{subfigure}
				\hspace{0.01\linewidth}
				\begin{subfigure}[c]{0.48\linewidth}
					\cfbox{neutral}{\includegraphics[width=\linewidth]{images/fdomain_powspec.eps}}
				\end{subfigure}
			\end{figure}
		\item \textbf{exercise}:
			\begin{itemize}
				\item examine spectra of different wave forms (sines, square, sawtooth, \ldots)
				\item examine spectral \textbf{frequency range}
				\item verify loss of \textbf{phase information} in (real valued) power spectra
			\end{itemize}
	\end{itemize}
\end{frame}

