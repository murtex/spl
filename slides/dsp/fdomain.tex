
	% frequency domain
\subsection{Frequency domain}

\begin{frame}[fragile]
	\frametitle{Frequency domain}
	\begin{itemize}
		\item \textbf{discrete Fourier transform}, time domain $\rightarrow$ frequency domain
			\begin{align*}
				X_k=\sum_{i=1}^Nx_i\,\ee^{-2\pi\ii\mfrac{(i-1)(k-1)}N}\in\mathbb C,\qquad k\in\{1,\ldots,N\}
			\end{align*}
			\begin{code}
>> Xk = fft( xi ) / N; \color{medium}% complex Fourier coefficients
			\end{code}
		\item $k$ is a \textbf{frequency index} (as $i$ was a time index)
			\begin{align*}
				k\rightarrow f_k=\frac{k-1}Nf_{\textrm S}
			\end{align*}
			\begin{code}
>> fk = (0:N-1) / N * fS; \color{medium}% frequency values
			\end{code}
		\item frequencies beyond Nyquist frequency are \textbf{negative frequencies}
			\begin{align*}
				f_k\rightarrow f_k=\begin{cases}f_k-f_{\textrm S}&\textrm{if $f_k>f_{\textrm{Ny}}$}\\f_k&\textrm{otherwise}\end{cases}
			\end{align*}
			\begin{code}
>> fk(fk > fNy) = fk(fk > fNy) - fS; \color{medium}% imply negative frequencies
			\end{code}
	\end{itemize}
\end{frame}

\begin{frame}[fragile]
	\frametitle{Frequency domain}
	\begin{itemize}
		\item \textbf{power spectral density} (also known as \textbf{power spectrum})
			\begin{align*}
				P_k=\lvert X_k\rvert^2\in\mathbb R\qquad\Leftarrow\qquad\sum_{k=1}^NP_k=P
			\end{align*}
			\begin{code}
>> Pk = abs( Xk ) .^ 2; \color{medium}% power spectral density
			\end{code}
		\item \textbf{real signals} ($x_i\in\mathbb R$)
			\begin{code}
>> Pk(fk < 0) = []; \color{medium}% remove negative frequency components
>> fk(fk < 0) = [];
			\end{code}
	\end{itemize}
\end{frame}

\begin{frame}
	\frametitle{Frequency domain}
	\begin{itemize}
		\item \textbf{example:} \texttt{matlab/fdomain.m}
			\begin{figure}
				\centering
				\begin{subfigure}[t]{0.48\linewidth}
					\includegraphics[width=\linewidth]{images/fdomain_decomp.eps}
				\end{subfigure}
				\hspace{0.01\linewidth}
				\begin{subfigure}[t]{0.48\linewidth}
					\includegraphics[width=\linewidth]{images/fdomain_powspec.eps}
				\end{subfigure}
			\end{figure}
	\end{itemize}
\end{frame}

\begin{frame}
	\frametitle{Frequency domain}
	\begin{itemize}
		\item spectral entropy
		\item inverse Fourier transform
	\end{itemize}
\end{frame}

